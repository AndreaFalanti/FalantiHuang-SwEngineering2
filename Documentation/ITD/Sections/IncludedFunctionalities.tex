For API endpoints specification, please refer to the .yaml file inside 'Server/other/api' folder, it provides the whole API structure following the OpenAPI3 standard conventions.

\begin{itemize}
	\item \textbf{Router:} implemented using \textit{oas-tools} middleware, adhering to OpenAPI3 specifications for API standardization. \textit{Oas-tools} reads and parse the .yaml file containing the API specifications and links the endpoints of the API to the respective handlers in the controller files, successfully providing request and response handling of the server. Response code are coherent to the HTTP status codes, so that the client can easily understand if something went wrong with the request or if it was successful.
	
	\item \textbf{LoginManager:} completely implemented, when the login data is valid, the server generates a session, storing the user's id, the account type and the city of the organization (in case of authorities). The clients receive the cookies to set in the header of the response, so that they can easily save them and insert them in the headers of successive request to identify themselves and access functionalities restricted by authorization. Other services can check the presence of user session and perform checks using the data contained in it.
	
	\item \textbf{SignUpManager:} implemented with some relaxations compared to RASD scenario's specifications. The component successfully allows the registration of both authorities and citizens to the service, performing checks that the email format is valid, the password has at least 6 characters and that the domain of the email is coherent to the account type. This last check means that the domain in authorities registrations must be the same of one of the registered organization, while for citizens it's true the vice versa. Authorities accounts are automatically associated to their organization by the domain check, satisfying the RASD and DD specifications.
	
	\item \textbf{OrganizationRegistrationManager:} implemented as explained in DD, adding also the possibility of adding a new city in the database, in case it's not present when registering the organization, so that the registration of the organization can be completed successfully. A prototype of the SafeStreets operator interface is accessible through browser to the URL '/pages/login.html', '/pages/organization.html', '/pages/city.html' and can be used successfully only by logged account associated to an organization with type 'system'. This type of organization can be inserted only by the system maintainer, performing the insertion directly into the database. Note that this interface needs actually major refinements, but can provide a simple showcase on how the interface scripting behaviour can be done.
	
	\item \textbf{ReportReceiverManager:} as described in the DD, allows the reception of a form containing the data of the violation from the client, completing it with the id of the submitter, the timestamp of the server when the report is received, and the place and city of the violation using reverse geocoding service API. \textit{Nominatim}, OpenStreetMap service API, has been used to get the place and city of the location provided in the report. For license plate recognition, clients can send a single photo to the apposite endpoint and the server will interrogate \textit{platerecognizer.com} OCR service to provide the license plates to clients. Note that as the free plan of the OCR service has been used for prototype, only one request per second can be made.
	
	\item \textbf{ReportVisualizationManager:} two endpoint are provided by the server, one is used to receive a list of reports based on the account type, the other for accessing a single report by id. Both endpoint checks who made the request by accessing the user session, then provide only the reports accessible by them. In case of the endpoint of report for id, it means that if a report is not accessible by the user, an error is returned directly. Citizens can't check the authorities who supervised the reports, as a privacy measure.
	
	\item \textbf{ReportValidationManager:} a single endpoint accessible only by authority accounts simply allows to change the status of a report. A preliminary check make sure that the report is accessible by the authority (same city of their organization), then launch a SQL update command on the report tuple with that id, changing status and supervisor id (obtained from session).
	
	\item \textbf{DataAnalysisManager:} as explained in the DD, the components allows the filtering of the reports by date, city and violation type; this can be done by appending specific query parameters to the endpoint URL. Data structure returned differs based on the account type that made the request. Another endpoint is also provided to fetch all cities registered to the system, so that a client can find easily the desired city id for completing the data analysis requests (for example, clients can use a dropdown to show all the cities and then use the id as the value to send).
\end{itemize}