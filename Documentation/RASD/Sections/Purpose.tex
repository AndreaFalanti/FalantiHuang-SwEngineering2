SafeStreets is a crowd-sourced application that intends to provide users with the possibility to notify authorities when traffic violations occur, and in particular parking violations. The application allows users to send pictures of violations, including their date, time, and position, to authorities. Examples of violations are vehicles parked in the middle of bike lanes or in places reserved for people with disabilities, double parking, and so on.

SafeStreets stores the violation reports provided by users, that can be made from the official mobile application or from the SafeStreets website, so that users and authorities can visualise and analyse the data received by the system, for example by highlighting the streets (or the areas) with the highest frequency of violations, or the vehicles that commit the most violations. The data can be accessed with different levels of visibility, where most sensitive data can only be mined by authorities.

SafeStreets also provide a service aimed at helping municipalities to identify potentially unsafe areas and suggest possible interventions. To have access to these features, the municipalities need to offer a service for retrieving info about accidents in their territory, so that Safestreets is able to cross a municipality's data with its own stored data and also analyse and suggest possible improvements to the areas with most reports.