\subsubsection{Definitions}
	\begin{itemize}
		\item \textbf{Citizen or common user}: a common citizen, without any public office, who uses the application to report traffic violations;
		\item \textbf{Authorities}: the public officials who certify the violations reported on the application;
		\item \textbf{Municipality}: the public institution that provides traffic violation data to the application, and may consult and analyze intervention suggestions from it;
		\item \textbf{Violation}: an event that does not conform with the traffic laws and that can be reported to authorities.
		\item \textbf{Intervention}: a possible public intervention produced by the application aimed at solving frequent violations in some areas of the city.
	\end{itemize}
\subsubsection{Acronyms}
\begin{itemize}
	\item GPS = Global Positioning System
	\item S2B = Software to Be
	\item RASD = Requirement Analysis and Verification Document
	\item PEC = Posta Elettronica Certificata
	\item API = Application Programming Interface
	\item AJAX = Asynchronous JavaScript and XML
	\item GDPR = General Data Protection Regulation
	\item OCR = Optical Character Recognition
	\item URI = Uniform Resource Identifier
\end{itemize}
\subsubsection{Abbreviations}
\begin{itemize}
	\item Gn = n-th goal;
	\item Dn = n-th domain assumption;
	\item Rn = n-th requirement.
\end{itemize}