\subsubsection{Definitions}
	\begin{itemize}
		\item \textbf{User}: a common citizen, without any public office, who uses the application to report traffic violations;
		\item \textbf{Authorities}: the public officials who certify the violations reported on the application;
		\item \textbf{Municipality}: the public institution that provides traffic violation data to the application, and may consult and analyze intervention suggestions from it;
		\item \textbf{Supervisor}: a person who verifies the validity of the violation reports sent to SafeStreets before sending them to the authorities, and also checks the identity of the registered authorities to avoid privacy safety;
		\item \textbf{Intervention}: a possible public intervention produced by the application aimed at solving frequent violations in some areas of the city.
	\end{itemize}
\subsubsection{Acronyms}
\begin{itemize}
	\item GPS = Global Positioning System
	\item S2B = Software to Be
	\item RASD = Requirement Analysis and Verification Document
	\item PCE = Public Certified Email
	\item API = Application Programming Interface
	\item AJAX = Asynchronous JavaScript and XML
	\item GDPR = General Data Protection Regulation
\end{itemize}
\subsubsection{Abbreviations}
\begin{itemize}
	\item Gn = n-th goal;
	\item Dn = n-th domain assumption;
	\item Rn = n-th requirement.
\end{itemize}