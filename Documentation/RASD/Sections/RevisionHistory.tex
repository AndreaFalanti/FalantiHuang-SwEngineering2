- Initial version 1.0 (10/11/2019)

- Version 1.1 (03/12/2019)
\begin{itemize}
	\item Add subsections to "Product perspective" section to improve readability.
	\item Add small note about user notification in "Report validation" state diagram. 
	\item Improve "Assumptions, dependencies and constraints" section structure and add paragraphs about "Terms and Conditions" and app permissions, improving also external API dependencies section.
	\item Add availability assumptions to OCR and maps service, forgotten in previous version.
	\item Add introduction to kinds of events, violation definition and OCR acronym.
	\item Delete username from citizen “registration” use case and delete definition of supervisor.
	\item Better link between use cases and user interface mockups, update the tabs’ names in use cases.
	\item Add and update mockups for municipality web app (used also in design document).
	\item Add possibility to load one or more photos in the “notify violation” use case.
	\item Update D5 to conform to final modeling of the software (also reflecting some changes in "Requirements" section).
	\item Improve "Product functions" section and better specify features explained in other parts of the documents, so that reader is aware from the beginning of that details.
	\item Small changes to "Generate report" state diagram and improvements to its description.
	\item Small improvements to "Suggestion evaluation" state diagram.
	\item Improve "user interfaces" section structure to identify more easily for which type of user the mockups are related.
	\item Revise goal logical derivation in "Requirements" section due to new changes.
	\item Change deployment target geographical area from Europe to Italy, because PEC was not used at European level and we could not verify the complete functionality of the system without faulty assumptions.  
	\item Add missing rate attribute in Intervention class of class diagram.
	\item Improve formatting of scenarios subsections in both user and third party sections.
	\item Correct typos.
\end{itemize}

- Version 1.2 (15/12/2019)
\begin{itemize}
	\item Add events related to confirmation emails to all user types use cases event flow, as they were not specified but are important to ascertain that the address is accessible by the user.
	\item Remove birthday from citizen's data.
	\item Update login mockup and signup (both citizen and authority) in "User interfaces" section.
	\item Remove vehicle class from class diagram, because the system doesn't focus on getting vehicle data besides the license plate and so it's not necessary. Also make photo attribute an array (multiple photos were introduced in version 1.1), change city attribute from "state" to "region", as the system is now designed to work in Italy.
	\item Remove vehicle entity and update worlds in alloy.
	\item Add new requirement and updated slightly other requirements and goals to change the registration method of authorities and municipalities from an automatic one based on PEC (that was unfeasible because Italian PEC register put both municipality and local police under same ISTAT category) to a new registration method where institutions take contact with SafeStreets operators directly to register their PEC domains and info. Only after this passage the workers of this organizations can be recognized by the system and register correctly. Another reason for the change was to avoid single persons to register to the system even if their organization was not interested in the service. Other sections of the document are also updated and expanded to reflect this change.
	\item Change Alloy model to conform to the changes explained in the previous point.
\end{itemize}