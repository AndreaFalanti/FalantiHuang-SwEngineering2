In the following sections the most important functions of the application are presented. The service of intervention suggestions to the municipalities is built on top of the storage data obtained by the reports from the citizens and optionally integrated with additional violation data provided by municipalities themselves.
\subsubsection{Violation Reporting}
This product function is the main feature of the system. After the citizen has signed up with an e-mail and a password, the user will be able to take some photos of the violation and upload them on the application, providing its violation type and optionally a description. The coordinates of the violation will be automatically determined using the GPS service of the user's device and filled in the report. After that, the user will be able to check the status of his/her report and will receive a notification when authorities validate or invalidate it. It's up to authorities to establish a proper way to validate or invalidate a report, as SafeStreets is only an intermediary between them and citizens. 
\subsubsection{Data Analysis}
The system also allows the registered users to mine data from its data storage, using different levels of authorization based on account type in order to protect the privacy of the individuals and avoid a misuse of the data from the common citizens. The pool of data mined will be restricted only to report validated by authorities to provide an accurate analysis of the real data about violations, avoiding contamination from erroneous or fraudulent reports.
\subsubsection{Intervention Suggestion generation}
This product function is offered when a municipality provides its own data about traffic accidents on its territory. After the municipality has signed up and optionally provided the API to access its data, SafeStreets will cross its own data gathered from the validated reports, find the possible interventions for the most afflicted areas of the territory and notify the municipality. Then, the municipality can log in and generate a list of suggested interventions, based on reports up to that time.

Because no standards are defined about violations' data cataloging, it would be too complex and time consuming for SafeStreets to build an interface to cross data with each municipality registered to the system that use a different data structure for storing violations. Because of this consideration, for using this functionality the municipality is responsible for complying to the conventions used by SafeStreets, that will be provided and accurately explained in the website page relative to the service.