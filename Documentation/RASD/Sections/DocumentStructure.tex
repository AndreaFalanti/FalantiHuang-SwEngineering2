The RASD document is composed by six chapters:

\textbf{Introduction}: first chapter of the document, it describes informally the main goals of the system and what problems they address. It also show an analysis of the events associated to the application domain, focusing in particular to shared phenomena (events that involve both users and the machine).

\textbf{Overall Description}: second chapter of the document, it focuses on describing accurately all the functionalities of the application, specifying also all the actors involved in the system. At the end of this section, assumptions on the world required for the correct behaviour of the system are clearly stated. 

\textbf{Specific Requirements}: third chapter of the document. First section describes all the interfaces required by the system, subdividing them in subgroups. Next section explicates all possible scenarios and user cases of all types of user. Other parts of the chapters specify in detail all the requirements that the system should satisfy, associating them with domain assumptions and goals to verify the completeness of the domain analysis. Last section of the chapter explains other constraints and limitation that apply to the system.

\textbf{Formal Analysis Using Alloy}: fourth chapter of the document. Contains Alloy model analysis, used to verify the correctness of the model described in previous section. 

\textbf{Effort Spent}: fifth chapter of the document, shows how much effort each member of the group has spent on the various chapters of the document.

\textbf{References}: final chapter, contains links to material, information or documentation related to the content discussed in the document.