The SafeStreets service is offered to common users to report traffic violations that hinder the normal flow of the traffic. It is thought to offer an aide to the public officers in detecting violations and thus provide a more regulated traffic system to the citizens. The service stands in the middle between the common citizen and the authorities, providing a real-time update of the situation on the streets.

The software will be distributed in Europe and will provide to any European citizen the possibility to report traffic violations by taking a photo of the transgressor's vehicle with its license plate visible and an appropriate view of the situation. The system will locate automatically the location of the violation, assuming the device of the user has the GPS service enabled. When the report is received by the system, it's stored into a database and authorities that have jurisdiction over that city can access it to consult its details and validate or invalidate it . The user will be able to track the status of their reports from the mobile app. SafeStreets, besides receiving reports from users, will also provide the possibility to consult its stored data with different levels of authorization for privacy safety reasons. A common citizen will be able to find the most frequent violations by location or time, but no private information of the violations will be provided. Whereas, the registered authorities will be able to perform the same queries and access to all the private information of the transgressors, for example, they will be able to find the most frequent transgressor in a certain area.

To help municipalities in identifying potentially unsafe areas, SafeStreets retrieves data about traffic accidents in their territories from their specific service. It crosses this data with its own stored data and computes possible interventions to areas most affected by violations, for example suggesting to add a barrier between the bike lane and the part of the road for motorized vehicles to prevent unsafe parking.

In the following we list the different kinds of events that can occur in the possible scenarios of interaction with SafeStreets. World events are phenomena occurring in the world that are not observable and not controllable by the SafeStreets' system. Shared events are phenomena that are either controlled by the world and observed by the SafeStreets's system or vice versa. Machine events are phenomena that are observed and controlled inside the SafeStreets' system, so they are not known to the world.

World events:
\begin{itemize}
    \item Traffic violations
    \item Violation detection
    \item Street interventions and improvements
    \item Authorities interventions
\end{itemize}

Shared events:
\begin{itemize}
    \item Violation report codification
    \item Data visualization and analysis
    \item Filtered data request
    \item Validated violations notification to authorities
    \item Intervention suggestion to municipality
\end{itemize}

Machine events:
\begin{itemize}
    \item Database queries
    \item Possible interventions computation
    \item License plate recognition
    \item Meta-data completion
\end{itemize}

\bigskip
\begin{center}
\begin{tabular}{|p{9cm}|l|l|}
	\hline 
	Phenomenon & Shared & Controlled by\\ 
	\hline 
	Traffic violations &  N & W \\ 
	\hline 
	Violation detection &  N & W\\ 
	\hline 
	Street interventions and improvements & N &  W \\ 
	\hline 
	Authorities interventions & N &  W \\
	\hline 
	Violation report codification & Y & W\\ 
	\hline 
	Data visualization and analysis & Y & M\\ 
	\hline
	Filtered data request & Y & W\\
	\hline 
	Validated violations notification to authorities & Y & M\\ 
	\hline 
	Intervention suggestion to municipality & Y & M\\ 
	\hline 
	Database queries & N & M\\ 
	\hline
	Possible interventions computation & N & M\\ 
	\hline
	License plate recognition & N & M\\ 
	\hline
	Meta-data completion & N & M\\ 
	\hline
\end{tabular} 
\end{center}
Legend:
\begin{itemize}
	\item Y := Yes, N := No.
	\item W := World, M := Machine.
\end{itemize}