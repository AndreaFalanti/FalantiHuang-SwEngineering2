Domain assumptions:
\begin{enumerate}[label={D\arabic*.}]
	\myitem {Registered emails must be unique.} \label{D_email}
    \myitem {Users report violations when they detect them.} \label{D_reports}
    \myitem {Traffic violations must occur regularly.} \label{D_occurrency}
    \myitem {Municipality service about accidents is always available.} \label{D_mun_availability}
   	\myitem {Municipality accidents' data is always accurate.} \label{D_mun_accuracy}
    \myitem {Report supervisor always validates the correctness of the reports.} \label{D_supervisor}
    \myitem {The GPS sensor of a user's device has an error of at most 5m from the real position.} \label{D_gps}
    \myitem {The internet connection is always available when the user interacts with the system.} \label{D_internet}
    \myitem {Every city has authority institutions.} \label{D_city_auth}
    \myitem {Municipality and authorities of each city have a PCE.} \label{D_pce}
    \myitem {Each nation's PCE service has an API to get the data of an institution from a PCE.} \label{D_pce_api}
\end{enumerate}

Any user who has a device with internet connection is able to access the services offered by SafeStreets (e.g. reporting violation with images and mining data about the previously reported violations). GPS sensor is required to report a violation.

Official apps supported mobile operating systems are Android (4.2 and above) and iOS. The website can be accessed with any modern browser with AJAX support.

The additional service of suggesting possible interventions to the most frequent traffic violations requires the municipality to provide an API to let SafeStreets access its data and augment it with its own for intervention analysis.

For image recognition and GPS localization services, the system employs external APIs to focus its main business on providing users the possibility to report traffic violations.

The S2B is assumed to be distributed anywhere on the globe; it needs only an email for the registration of common citizens, and for authorities ???

[TODO: GDPR ?]