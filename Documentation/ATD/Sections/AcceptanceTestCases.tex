This section is mainly focused on testing the fulfillment of functionalities involving the use cases described in the RASD(section 3.2.2). Testing has been performed on a physical smartphone device running Android 10 (Q) and an Android emulator running Android 8.1 (Oreo).

\paragraph{Use case: Account Registration}
This test deals with the event flow of the registration of an account. From the homepage, we have clicked the registration button and filled all the fields in the registration page. We have found in the registration form a button that opens a map to retrieve the region, which was not described in the event flow of the use case, and found difficulties to understand how to actually retrieve the region from the map. Only by contacting the authors we were able to understand how to retrieve the region, by long-pressing on a location of the map and returning back to previous screen using "back" button. There's no description about the implementation of how an authority is recognized as such in the ITD document, so we have no means to test it. We tried to register multiple accounts with the same username but the test failed the exception described in the RASD. It seems more like it is the email that must be unique.
\paragraph{Use case: Log In} 
This test concerns the log in screen. We have tested some random non-registered data to see if we could access the application directly, and it has failed the access as it is supposed to work. The case of an existing email but mismatched password also have worked as described, it asked to refill the log in form. One small issue is that on screen UI its stated that the username must be inserted in the first field, but the email is required instead.
\paragraph{Use case: Violation Report} 
This test deals with the functionalities of the violation report screen. We have tried to take a photo without the license plate and the app correctly informed the us that the license plate was not recognized and does not allow us to send the report. While with a valid photo, we were able to send the report with the selected violation report and additional information in the "Further details" field.  There is an issue with the "Further details" field that is not correctly saved or received, because it's not displayed in the violation details. There is also one visualization issue after the report is sent, after returning to report tab, the report list is not updated with the new one, also no message about of the success of the operation is shown on screen. To actually check that the report is received, we found that we could refresh by swiping down, but an automatic refresh could be more elegant.
\paragraph{Use case: Information Request} 
This test performs some checks on the information request screen. We have tried different combinations of values for the fields, and the requested data was correctly retrieved. Even though the application allows a combination of a "from" field date that comes later in time than the "to" field date, the data queried has no problem and it's correct. When no reports are found, a black screen with map visualization on Rome is displayed, the interface could be improved in this case to notify the absence od violations.
\paragraph{Use case: Authority-Request}
This test concerns the additional functionalities provided to an authority user. We have registered an authority account and successfully logged into the application. We have tried the additional functionality that allows an authority to check reports in their area of jurisdiction and the correct data was retrieved (all reports in its region). For explore tab, license plate must be uppercase, otherwise it won't match the violations correctly. Also, it seems that authority see the same information of a normal user, we don't know for certain if it's by design or something is missing.