This section is mainly focused on testing the fulfillment of functionalities involving the use cases described in the RASD(section 3.2.2).
\paragraph{Use case: Account Registration}
This test deals with the event flow of the registration of an account. From the homepage, we click the registration button and fill all the fields in the registration page. We find a map button to retrieve the region in the registration form which was not described in the event flow of the use case and find difficulties to understand how to retrieve the region from the map. Only by contacting the authors we were able to understand how to retrieve the region, by long-pressing on a location of the map. There's no description about the implementation of how an authority is recognized as such in the ITD document, so we have no means to test it. We tried to register multiple accounts with the same username but the test failed the exception described in the RASD. It seems more like it is the email that must be unique.
\paragraph{Use case: Log In} 
This test concerns the log in screen. We test some random non-registered data to see if we can access the application directly, and it fails the access as it is supposed to work. The case of an existing email but mismatched password also works as described, it asks to refill the log in form. Actually on screen its stated that the username must be inserted in the field, but instead the email is required.
\paragraph{Use case: Violation Report} 
This test deals with the functionalities of the violation report screen. We try to take a photo without the license plate and the app correctly informs the us that the license plate was not recognized and does not allow us to send the report. While with a valid photo, we are able to send the report with the selected violation report and additional information in the "Further details" field. There is one visualization issue after the report is sent, after returning to report tab the report list is not updated with the new one, also no message about of the success of the operation is shown on screen. To actually check that the report is received, we find that we could refresh by swiping down, but an automatic refresh could be more elegant. Another issue is that "Further details" field is not correctly saved or received, so its not displayed in the violation details.
\paragraph{Use case: Information Request} 
This test performs some checks on the information request screen. We try different combinations of values for the fields, and the requested data is correctly retrieved. Even though the application allows a combination of a from field date that comes later in time than the to field date, the data queried has no problem and is correct. When no reports are found, a black screen with map visualization on Rome is displayed, the interface could be improved in this case to notify the absence od violations.
\paragraph{Use case: Authority-Request}
This test concerns the additional functionalities provided to an authority user. We register an authority account and successfully log in the application. We try the additional functionality that allows an authority to check reports in their area of jurisdiction and the correct data is retrieved. For explore tab, license plate must be uppercase, otherwise it won't match the violations correctly. Also, it seams that authority see the same information of a normal user.