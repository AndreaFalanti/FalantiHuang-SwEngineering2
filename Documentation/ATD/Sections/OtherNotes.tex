Regarding the client versions, in RASD section 3.5.6 it's stated that the application must work on the most available browsers, but neither in installation steps neither in DD there are information about a web application, maybe it was designed for municipalities and therefore is not present as it was not required in base service but we have no other clues.

We found some inconsistencies between the mockups and the actual screen of the application, especially in the main menu, that is completely different. Also the interfaces are a bit cryptic to navigate, especially completing the fields where is required to click on the bottom next to field to open a map and then long press a location and manually click the back button to return to the previous screen. These steps are really unconventional and no information about what to do is shown on screen, we actually have to mail the authors to understand how to proceed for the map interactions.

It seems that the client works fine on multiple versions of Android, but we have no way to test it on iOS. RASD section 3.5.6 state that must work for the most known smartphone operative systems, so also a way of installing on iOS should have been provided.

We have an hard time in searching important information about the system to actually provide a valid testing. Even if ITD is good structured and describe well the implementation choices, RASD and DD documents are pretty vague in many sections, for example in no sections is explained accurately what special permissions an authority account actually has compared to a normal user account. This is stated vaguely in some uses cases and the introduction by the "Privileged request" definition (RASD section 1.3.1), but we have no idea of what fields should be accessible only from authority and therefore couldn't test that the requirement is satisfied (actually the requirement of different info on data request isn't even stated in RASD requirements section 3.2.4).