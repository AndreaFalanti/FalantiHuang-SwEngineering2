\subsubsection{Three tier architecture}
The system is developed following a classical three tier architecture, composed by presentation layer, application layer and data layer. This type of architecture is a common standard nowadays for medium size applications, because it allows to distribute work load between three tier of machines, decoupling the tiers neatly without compromising the functionalities. Every tier can be modified internally in a flexible way, without compromising the functionalities of other tiers (only the changes to interfaces between tiers are critical). Another important quality of this architecture style is the fact that data tier is accessible only from application server, so the system is more secure because clients are unable to communicate directly with the database.

\subsubsection{RESTful API}

The clients interact and send requests to the SafeStreets servers to a REST API, that provides a web interface common between all the clients. Using a REST API allows to have a de-coupling between clients and servers functionalities. The clients only need to call the correct endpoint for retrieving data or perform an action, expecting a certain output structure but without knowing the computations that happen behind the curtains. This also helps the maintainability of both servers and clients, because server can be updated without affecting directly the clients if the API endpoints are not modified (but can be expanded).

To achieve these goals, a REST API need to follow these properties:
\begin{itemize}
	\item \textbf{Uniform interface:} API URLs have similar structure and query attributes, so that it's easier to understand how to use the API interface and what each endpoint does.
	\item \textbf{Stateless:} API returned data by a same endpoint has always the same structure, the server is not aware of the specific state scope of the call or the state of the client, it just provides data that can be used by the client as they please.  
	\item \textbf{Client-Server model:} client and server evolve in a separate way and communicate only through API endpoints.
	\item \textbf{Caching:} a client-side or server-side cache system can be used to improve performances and speed up interactions, especially in case of multiple same requests in a very short period of time.
	\item \textbf{Layered architecture:} the system uses a standard 3-tier architecture. Clients communicate with the servers through API calls, where specific backend code handle each request. These handlers call the managers that have the functions used to interrogate and manipulate the data layer, to fetch the data that clients require to display or perform specific actions, like login.
\end{itemize}