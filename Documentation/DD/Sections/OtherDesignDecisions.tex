\subsubsection{Relational database}
A relational database is chosen to store the data managed by the system, as the data can be structured with a fixed schema without the necessity of "optional" field that can be frequently null and clutter the database. Another reason to choose a SQL database instead of a noSQL one is that SQL is a mature and optimized query language that the team is acquainted with and knows how to optimize big queries required for data analysis functionality; using a relational database with proper indexing should guarantee fast and reactive performance for every operation. One disadvantage of the usage of a relational database is that horizontal scaling is really hard to perform and so vertical scaling is the primary choice in case the system need more power for handling its data, while noSQL can scale horizontally in an easy way and have less upgrading costs.

\subsubsection{Asynchronous programming over synchronous multi-threaded approach}
Application servers should prioritize an asynchronous programming model over a multi-threaded approach. Asynchronous approach maximize throughput and scalability of the web server, because allows multiple I/O operation on a single thread, so that there is no thread context switching overhead. In this type of architecture, the system can perform all operations without blocking the thread, because it can pause tasks and perform callbacks when a task required for another one is terminated, performing other operations in the meantime. Note that asynchronous programming could also work on multiple threads to optimize the CPU utilization of the server, but the various task are combined in an efficient way on each thread and multiple client requests can be executed on each one thanks to the non blocking approach, instead in a more classical synchronous multi-thread each task that performs I/O need a thread, degenerating quickly in a huge amount of threads in case of multiple request in a small time.