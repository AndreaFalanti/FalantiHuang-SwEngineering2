[TODO: define type of implementation (bottom-up)]

From the previous analysis it emerges that the components that are more difficult to implement are the ones relative to the aggregated data visualization and intervention list generation functionalities. These components would require the higher amount of testing in order to assure that the system will work flawlessly with the minimal amount of bugs. [TODO: add discrimination based on relevance of the functionality]

As first step of the implementation of SafeStreets system, the underlying database must be created. A small database populated with some dummy data should also be developed, following the same structure of the production one. This database will allow the team to test easily the API of the server, because the output of a query will be always known if after each test any modification done on this database is reversed, so that it remains untouched.

After database implementation, the next step is to develop the server API. Defining clearly the API endpoints, the input they receive and the output they produce will allow to start building the various components that handles the logic behind the API calls, based on the received parameters. Also clients could be developed in parallel as they only display data and call the defined endpoints, they only need to satisfy the required input defined in API endpoints and working on manipulating the data that is expected back, which structure is also defined by the API definition (along with possible errors). Because of this considerations, clients and server can be developed and tested in parallel by two separate teams, that have competence with the respective technologies of these subsystems.

Concerning the application server, based on the relevance and complexity of the functionality, the following order of implementation should be followed to minimize the effort and build a cohesive system. For each functionality, an high-level implementation guideline is also provided to give a general idea of the system to the developers' team and interested stakeholders:
\begin{itemize}
	\item \textbf{Login and registration:} standard in all types of web applications, this functionality should be pretty fast to implement because of already existing plugins and libraries. The only difficulty in the implementation is caused by the automatic recognition of municipalities and authorities performed by the components, done by querying the Italian public PEC register (\url{https://www.indicepa.gov.it/documentale/index.php}) [TODO: explain better how]. Implementing this component will allow to exploit the session, registered in the server after each client sign in, also in tests, important in about all functionalities because accessible data and services differs based on account type.
	\item \textbf{Report submission:} this functionality it's essential for the whole system, without the reports sent by the user all others functionalities would be useless because of the absence of data. Considering this fact, this functionality need to be implemented first in the server and exhaustively tested to check that valid report are received, completed and stored in the database correctly, while invalid one are refused returning a proper error, following API definition. This component is also tied to the external OCR service, so it's important to choose a quality service with an easily accessible interface for performing the license plate recognition.
	\item \textbf{Report detail visualization:} this functionality is pretty simple to implement server-side and not time consuming, because the component just need to check if the requested report is accessible from the client or return to client a list of accessible reports to check then individually. This can be done respectively by checking what type of account made the request, by checking the active sessions database, then adapt the query to perform on the main database based on the account info for retrieving the accessible list of reports, instead for single report access the component just needs to check with some predicates if the client can retrieve that report. Moreover, this functionality is essential for authorities registered to the system, as their main focus is to check the reports and their validity to efficiently patrol the city and punish the transgressors, so it's advisable to implement it earlier.
	\item \textbf{Intervention generation:} one of the most complex functionality of the system, shouldn't be implemented to late because it's the target service for a whole type of stakeholders (municipalities) and because of this needs proper testing before software release. Both InterventionManager and MunicipalityDataIntegrationManager components are involved in this functionality realization and need to coordinate together to reach the goal. Firstly, MunicipalityDataIntegrationManager must be implemented, because it's used by InterventionManager before starting the main computation of the interventions. Data integration is performed through the municipality's API that made the request for interventions, so the account info must be retrieved through checks on session and main databases. If the API is consistent with the guidelines defined by SafeStreets, the component will be able to retrieve the data and then integrate it in the database with an insert operation, otherwise it will return an error. In either cases, the InterventionManager starts the algorithm for finding the interventions relative to the municipality's area. The discriminating factors for defining consistent and valid interventions are the density of the violations in a relative small area, then the presence of a certain percentage of a violation type or a combination of them, these two piece of info allows the possibility to define sensible intervention for the analyzed situation.  
\end{itemize}