During the implementation, given the relatively small components we apply a bottom-up test strategy. The system's components must be progressively tested and integrated together, but first of all each component must be unit tested, so that it's possible to recognize if a component is bugged or faulty. Integration tests will be performed after unit tests, providing the certainty that the flow of messages through the interfaces and the final output of the operations are consistent with the expected results.

As mentioned before, a small database with fictitious data should be implemented in first steps of implementation, so that it's possible to test the component more easily and as soon as possible to discover any possible issues. It's important that tests revert the operations performed on this test database after the test is completed, otherwise the modification will alter the results of successive runs of test suites.

Once the whole system is integrated, a final system test must be performed to verify the functional and non-functional requirements of the system. In particular, given the possibility of huge requests on the servers when the system is deployed, it is important to carry out a performance testing to identify bottlenecks affecting [response time, utilization, throughput].