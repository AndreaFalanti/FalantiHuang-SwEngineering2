During the implementation, given the relatively small components we apply a bottom-up test strategy. Following the bottom-up definition, we will start by unit testing the components of the application server that don't need stabs for their functionalities, so that we can use implement only the drivers for simulating interactions with these components. When the components pass their relative tests, it's possible to start the implementation of the modules that use that components so that we can substitute them to the drivers for integration tests. Therefore system's components must be progressively tested and integrated together, but, first of all, each component feature that doesn't rely on other components must be unit tested, so that it's possible to recognize if a single component is bugged or faulty. Integration tests will be performed after unit tests, providing the certainty that the flow of messages through the interfaces and the final output of the operations are consistent with the expected results.

As mentioned before, a small database with fictitious data should be implemented in first steps of implementation, so that it's possible to test the components more easily and to discover as soon as possible any issue. It's important that tests revert the operations performed on this test database after the test is completed, otherwise the modifications will alter the results of successive runs of test suites.

In the application server, the most important tests are relative to transactions done on the database and the data retrieved from database with the queries, these tests assure that from the data passed from the API request is possible to retrieve or modify the data correctly with the implemented query command. After these tests, it's also important to test the final output of the operations done by each component, to check if the output is consistent with the returned data schema of the API and that the computation output reflects the results expected by the requirements of the system.

In the clients a similar approach will be followed. It will be tested that from the UI interface the data inserted by the user is consistent with the data displayed on screen (its input) and that it's correctly encoded following the target API endpoint schema. Other test suites instead will assure that when receiving some data reflecting the API output schema, the client will be able to display this data correctly on the user screen. 

Once the whole system is integrated, a final system test must be performed to verify the functional and non-functional requirements of the system. In particular, given the possibility of huge requests on the servers when the system is deployed, it is important to carry out a performance testing to identify bottlenecks affecting response time, utilization and throughput.