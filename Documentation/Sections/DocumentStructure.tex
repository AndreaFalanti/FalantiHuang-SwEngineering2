The Design Document is composed by seven chapters:

\textbf{Introduction:} first chapter of the document, it describes informally the main goals of the system and what problems they address. Note that first two sections are borrowed from RASD document, as they provide a general introduction to the system.

\textbf{Architectural design:} second chapter, it provides information about how the system architecture, how it is physically deployed and which technologies are used to fulfill its functionalities. It also describes the patterns and design decision used to implement the project and why they are useful in achieving goals of the system.

\textbf{User interface design:} third chapter, it contains the mockup of the screens visualized by clients that interact with the system, already defined in RASD and reported here for simplicity of reading.

\textbf{Requirement traceability:} fourth chapter, it describes explicitly how each requirement of the system is fulfilled and which components are involved for its realization.

\textbf{Implementation, integration and testing:} fifth chapter, it provides information about how the system will be implemented, how all its components will be integrated to build the definitive system and how testing will be performed. To achieve this, a preliminary analysis its performed to identify the importance and difficulty of implementation of the various functionalities and annexed components. The results of the analysis are used to establish the order of implementation of the components and how much testing will be performed on them.

\textbf{Effort spent:} sixth chapter of the document, shows how much effort each member of the group has spent on the various chapters of the document.

\textbf{References:} final chapter, contains links to material, information or documentation related to the content discussed in the document.


